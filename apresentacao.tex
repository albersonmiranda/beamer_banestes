% Options for packages loaded elsewhere
\PassOptionsToPackage{unicode}{hyperref}
\PassOptionsToPackage{hyphens}{url}
%
\documentclass[
  ignorenonframetext,
]{beamer}
\usepackage{pgfpages}
\setbeamertemplate{caption}[numbered]
\setbeamertemplate{caption label separator}{: }
\setbeamercolor{caption name}{fg=normal text.fg}
\beamertemplatenavigationsymbolsempty
% Prevent slide breaks in the middle of a paragraph
\widowpenalties 1 10000
\raggedbottom
\setbeamertemplate{part page}{
  \centering
  \begin{beamercolorbox}[sep=16pt,center]{part title}
    \usebeamerfont{part title}\insertpart\par
  \end{beamercolorbox}
}
\setbeamertemplate{section page}{
  \centering
  \begin{beamercolorbox}[sep=12pt,center]{part title}
    \usebeamerfont{section title}\insertsection\par
  \end{beamercolorbox}
}
\setbeamertemplate{subsection page}{
  \centering
  \begin{beamercolorbox}[sep=8pt,center]{part title}
    \usebeamerfont{subsection title}\insertsubsection\par
  \end{beamercolorbox}
}
\AtBeginPart{
  \frame{\partpage}
}
\AtBeginSection{
  \ifbibliography
  \else
    \frame{\sectionpage}
  \fi
}
\AtBeginSubsection{
  \frame{\subsectionpage}
}

\usepackage{amsmath,amssymb}
\usepackage{lmodern}
\usepackage{iftex}
\ifPDFTeX
  \usepackage[T1]{fontenc}
  \usepackage[utf8]{inputenc}
  \usepackage{textcomp} % provide euro and other symbols
\else % if luatex or xetex
  \usepackage{unicode-math}
  \defaultfontfeatures{Scale=MatchLowercase}
  \defaultfontfeatures[\rmfamily]{Ligatures=TeX,Scale=1}
\fi
% Use upquote if available, for straight quotes in verbatim environments
\IfFileExists{upquote.sty}{\usepackage{upquote}}{}
\IfFileExists{microtype.sty}{% use microtype if available
  \usepackage[]{microtype}
  \UseMicrotypeSet[protrusion]{basicmath} % disable protrusion for tt fonts
}{}
\makeatletter
\@ifundefined{KOMAClassName}{% if non-KOMA class
  \IfFileExists{parskip.sty}{%
    \usepackage{parskip}
  }{% else
    \setlength{\parindent}{0pt}
    \setlength{\parskip}{6pt plus 2pt minus 1pt}}
}{% if KOMA class
  \KOMAoptions{parskip=half}}
\makeatother
\usepackage{xcolor}
\newif\ifbibliography
\setlength{\emergencystretch}{3em} % prevent overfull lines
\setcounter{secnumdepth}{-\maxdimen} % remove section numbering


\providecommand{\tightlist}{%
  \setlength{\itemsep}{0pt}\setlength{\parskip}{0pt}}\usepackage{longtable,booktabs,array}
\usepackage{calc} % for calculating minipage widths
\usepackage{caption}
% Make caption package work with longtable
\makeatletter
\def\fnum@table{\tablename~\thetable}
\makeatother
\usepackage{graphicx}
\makeatletter
\def\maxwidth{\ifdim\Gin@nat@width>\linewidth\linewidth\else\Gin@nat@width\fi}
\def\maxheight{\ifdim\Gin@nat@height>\textheight\textheight\else\Gin@nat@height\fi}
\makeatother
% Scale images if necessary, so that they will not overflow the page
% margins by default, and it is still possible to overwrite the defaults
% using explicit options in \includegraphics[width, height, ...]{}
\setkeys{Gin}{width=\maxwidth,height=\maxheight,keepaspectratio}
% Set default figure placement to htbp
\makeatletter
\def\fps@figure{htbp}
\makeatother

% packages, font, color, and new commands
\usepackage{amsfonts, amsmath, oldgerm, lmodern, animate}
\usepackage{verbatim, ragged2e}
\usepackage{bm}
\usepackage{tema/beamerthemesintef}
\usefonttheme{serif}

% justificar
\justifying

% TYPESETTING ELEMENTS

% style of section presented in the table of contents
\setbeamertemplate{section in toc}{$\blacktriangleright$~\inserttocsection}

% style of subsection presented in the table of contents
\setbeamertemplate{subsection in toc}{}
% \setbeamertemplate{subsection in toc}{\footnotesize\hspace{1.2 em}$\blacktriangleright$~\inserttocsubsection}

% automate subtitle of each frame
\makeatletter
    \pretocmd\beamer@checkframetitle{\framesubtitle{\thesection \, \secname}}
\makeatother

% avoid numbering of frames that are breaked into multiply slides
\setbeamertemplate{frametitle continuation}{}

% at the begining of section, add table of contents with the current section highlighted
\AtBeginSection[]
{
    \begingroup
    \setbeamertemplate{footline}{}
    \themecolor{blue}
    \begin{frame}{SUMÁRIO}
        \tableofcontents[currentsection]
    \end{frame}
    \endgroup
}

% at the beginning of subsection, add subsection title page
\AtBeginSubsection[]
{
    \begin{frame}{\,}{\thesection \, \secname}
        \fontfamily{ptm}\selectfont
        \centering\textsl{\textbf{\textcolor{sintefblue}{
            \large Section \thesection.\thesubsection
            \vskip15pt
            \LARGE \subsecname
        }}}
    \end{frame}
}

% code bolck setting
\definecolor{codegreen}{RGB}{101,218,120}
\definecolor{codegray}{rgb}{0.5,0.5,0.5}
\definecolor{codepurple}{rgb}{0.58,0,0.82}
\definecolor{backcolour}{rgb}{0.95,0.95,0.92}

\lstdefinestyle{mystyle}{
    % backgroundcolor=\color{backcolour},
    commentstyle=\color{airforceblue},
    keywordstyle=\color{magenta},
    numberstyle=\tiny\color{codegray},
    stringstyle=\color{codepurple},
    basicstyle=\ttfamily\scriptsize,
    breakatwhitespace=false,
    breaklines=true,
    captionpos=b,
    keepspaces=true,
    numbers=left,
    numbersep=5pt,
    showspaces=false,
    showstringspaces=false,
    showtabs=false,
    tabsize=4,
    xleftmargin=10pt,
    xrightmargin=10pt,
}

\lstset{style=mystyle}

% NEW COMMANDS

% set colored hyperlinks command
\newcommand{\hrefcol}[2]{\textcolor{airforceblue}{\href{#1}{#2}}}
\newcommand{\hlinkcol}[1]{\hrefcol{#1}{#1}}


% centering paragraph statement
\newcommand{\centerstate}[1]{
    \centering
    \begin{columns}
        \begin{column}{0.8\textwidth}
            #1
        \end{column}
    \end{columns}
}

% colored textbf
\newcommand{\ctextbf}[1]{\textbf{\textcolor{sintefblue}{#1}}}
\newcommand{\btextbf}[1]{\textbf{\textcolor{airforceblue}{#1}}}

% colored textsl
\newcommand{\ctextsl}[1]{\textsl{\textcolor{sintefblue}{#1}}}
\newcommand{\btextsl}[1]{\textsl{\textcolor{airforceblue}{#1}}}

% colored emph
\newcommand{\cemph}[1]{\emph{\textcolor{sintefblue}{#1}}}
\newcommand{\bemph}[1]{\emph{\textcolor{airforceblue}{#1}}}

% about page
\newcommand{\aboutpage}[2]{
    \begingroup
    \setbeamertemplate{footline}{}
    \themecolor{blue}
    \begin{frame}[c]{#1}{\,}
        \centering
        \begin{minipage}{\textwidth}
            \usebeamercolor[fg]{normal text}
            \centering
            \Large \textsl{\normalsize #2}
        \end{minipage}
    \end{frame}
    \endgroup
}

% bibliography page
\newcommand{\bibliographpage}{
    \section{References}

    \begingroup
    \themecolor{blue}
    \begin{frame}[allowframebreaks]{References}{\,}
        \tiny
        \printbibliography[heading=none]
    \end{frame}
\endgroup
}
\makeatletter
\makeatother
\makeatletter
\makeatother
\makeatletter
\@ifpackageloaded{caption}{}{\usepackage{caption}}
\AtBeginDocument{%
\ifdefined\contentsname
  \renewcommand*\contentsname{Table of contents}
\else
  \newcommand\contentsname{Table of contents}
\fi
\ifdefined\listfigurename
  \renewcommand*\listfigurename{List of Figures}
\else
  \newcommand\listfigurename{List of Figures}
\fi
\ifdefined\listtablename
  \renewcommand*\listtablename{List of Tables}
\else
  \newcommand\listtablename{List of Tables}
\fi
\ifdefined\figurename
  \renewcommand*\figurename{Figure}
\else
  \newcommand\figurename{Figure}
\fi
\ifdefined\tablename
  \renewcommand*\tablename{Table}
\else
  \newcommand\tablename{Table}
\fi
}
\@ifpackageloaded{float}{}{\usepackage{float}}
\floatstyle{ruled}
\@ifundefined{c@chapter}{\newfloat{codelisting}{h}{lop}}{\newfloat{codelisting}{h}{lop}[chapter]}
\floatname{codelisting}{Listing}
\newcommand*\listoflistings{\listof{codelisting}{List of Listings}}
\makeatother
\makeatletter
\@ifpackageloaded{caption}{}{\usepackage{caption}}
\@ifpackageloaded{subcaption}{}{\usepackage{subcaption}}
\makeatother
\makeatletter
\@ifpackageloaded{tcolorbox}{}{\usepackage[many]{tcolorbox}}
\makeatother
\makeatletter
\@ifundefined{shadecolor}{\definecolor{shadecolor}{rgb}{.97, .97, .97}}
\makeatother
\makeatletter
\makeatother
\ifLuaTeX
  \usepackage{selnolig}  % disable illegal ligatures
\fi
\IfFileExists{bookmark.sty}{\usepackage{bookmark}}{\usepackage{hyperref}}
\IfFileExists{xurl.sty}{\usepackage{xurl}}{} % add URL line breaks if available
\urlstyle{same} % disable monospaced font for URLs
\hypersetup{
  pdftitle={KEY PERFORMANCE INDICATORS},
  pdfauthor={Alberson Miranda},
  hidelinks,
  pdfcreator={LaTeX via pandoc}}

\title{KEY PERFORMANCE INDICATORS}
\subtitle{REVISÃO DE LITERATURA PARA IFs}
\author{Alberson Miranda}
\date{dezembro de 2022}

\begin{document}
\frame{\titlepage}
\ifdefined\Shaded\renewenvironment{Shaded}{\begin{tcolorbox}[breakable, borderline west={3pt}{0pt}{shadecolor}, boxrule=0pt, sharp corners, frame hidden, interior hidden, enhanced]}{\end{tcolorbox}}\fi

\hypertarget{por-que-usar-kpis}{%
\section{POR QUE USAR KPIs?}\label{por-que-usar-kpis}}

\begin{frame}{O QUE SÃO KPIs}
\protect\hypertarget{o-que-suxe3o-kpis}{}
Key Performance Indicators (KPI) são métricas financeiras e não
financeiras usadas para ajudar uma organização a definir e medir seu
progresso na direção de objetivos organizacionais (Nimalathasan, 2009)
\end{frame}

\begin{frame}{BENEFÍCIOS}
\protect\hypertarget{benefuxedcios}{}
Parmenter (2015) coloca três benefícios principais na definição dos
fatores de sucesso e suas medidas de performance associadas:

\begin{enumerate}
\tightlist
\item
  Clareza de propósito, do alinhamento diário até os fatores de sucesso
  organizacionais
\item
  Melhoria de performance através do foco. Menos indicadores, porém mais
  significativos
\item
  Cria mentalidade de dono, promove empoderamento e sentimento de
  realização em todos os níveis da organização
\end{enumerate}
\end{frame}

\begin{frame}{BENEFÍCIOS}
\protect\hypertarget{benefuxedcios-1}{}
LEPOLD, TANZER \& JIMÉNEZ (2018) concluem que a satisfação do empregado
tem como dois de seus preditores:

\begin{enumerate}
\tightlist
\item
  autoeficácia (crença do indivíduo em sua própria capacidade e
  habilidade no seu domínio específico de trabalho)
\item
  expectativas de influência (expectativas do indivíduo de ser capaz de
  influenciar os KPIs).
\end{enumerate}
\end{frame}

\begin{frame}{BENEFÍCIOS}
\protect\hypertarget{benefuxedcios-2}{}
O engajamento é essencial para a produtividade da empresa e pode ser
decomposto em três fatores (LEPOLD, TANZER \& JIMÉNEZ, 2018):

\begin{enumerate}
\tightlist
\item
  Vigor: altos níveis de energia e resiliência mental durante o
  trabalho, disposição em colocar esforço numa atividade e persistência
  em face de dificuldades
\item
  Dedicação: reflete envolvimento e identificação psicológica de um
  indivíduo por seu trabalho, traduzido por um senso de significância,
  entusiasmo, inspiração, orgulho e desafio
\item
  Absorção: descreve a extensão da concentração total e absorção de
  alguém em seu trabalho, o sentimento de ``tempo voar''
\end{enumerate}
\end{frame}

\hypertarget{seleuxe7uxe3o-de-kpis}{%
\section{SELEÇÃO DE KPIs}\label{seleuxe7uxe3o-de-kpis}}

\begin{frame}{KPIs NA LITERATURA}
\protect\hypertarget{kpis-na-literatura}{}
Sejam quais forem os KPIs selecionados, eles devem refletir os fatores
de sucesso essenciais de uma empresa e estar conectados à estratégia
organizacional (Nimalathasan, 2009).
\end{frame}

\begin{frame}{KPIs NA LITERATURA}
\protect\hypertarget{kpis-na-literatura-1}{}
Bancos que atuam focados em KPIs de estratégia voltada para valorização
da ação da empresa e pagamento de dividendos têm em média 5,8 pp a mais
de Retorno Total do Investidor (TSR) que aqueles que focam em outros
indicadores, como o Retorno sobre o Patrimônio (ROE), também os
superando em lucratividade, crescimento e liquidez (SCHMALTZ, LUEG \&
AGERHOLM, 2019)
\end{frame}

\begin{frame}{KPIs NA LITERATURA}
\protect\hypertarget{kpis-na-literatura-2}{}
Daryakin et al.~(2019) elencam 6 fatores que separam os KPIs efetivos
dos desnecessários:

\begin{enumerate}
\tightlist
\item
  Unificados: KPIs devem coincidir com os objetivos estratégicos e
  marcos da empresa
\item
  Factíveis: KPIs devem ser mensuráveis e fáceis de se obter. Do
  contrário, não há porquê usá-los
\item
  Coerentes: KPIs devem coincidir com as atividades das áreas. Ex.: faz
  sentido medir índice de inadimplência por gerente?
\item
  Exatos: KPIs devem ser precisos e confiáveis.
\item
  Inteligentes: KPIs devem permitir insights sobre a forma de trabalho
  da área e o que deve ser feito.
\item
  Dinâmicos: KPIs devem estar em constante variação. Não faz sentido
  acompanhar um número que não se altera.
\end{enumerate}
\end{frame}

\begin{frame}{KPIs NA LITERATURA}
\protect\hypertarget{kpis-na-literatura-3}{}
Wu (2012) elenca alguns KPIs em função da sua influência causal nos
demais indicadores observados, o que quer dizer que trabalhando nesses
indicadores obtém-se efeitos generalizados na instituição bancária.

\begin{enumerate}
\tightlist
\item
  Experiência do Cliente {[}também indicado em Nimalathasan (2009){]}
\item
  Performance de Vendas
\item
  Taxa de Retenção de Cliente
\item
  Market Share
\item
  Performance Gerencial
\end{enumerate}
\end{frame}

\begin{frame}{KPIs NA LITERATURA}
\protect\hypertarget{kpis-na-literatura-4}{}
Para instituições bancárias, as métricas não financeiras,
particularmente as da perspectiva do cliente, devem ser mais
enfatizadas.

A orientação voltada ao cliente, estabelecendo as métricas relacionadas
ao cliente, deve estar em primeiro plano na definição dos objetivos
estratégicos da empresa.
\end{frame}

\begin{frame}{REFERÊNCIAS}
\protect\hypertarget{referuxeancias}{}
\footnotesize

DARYAKIN, A. A.; SKLYAROV, A. A.; KHASANOV, K. A. The Role of Key
Performance Indicators (KPI) in Banking Activities. \textbf{Gênero \&
Direito}, v. 8, n.~4, 2 out. 2019.

LEPOLD, A.; TANZER, N.; JIMÉNEZ, P. Expectations of Bank Employees on
the Influence of Key Performance Indicators and the Relationship with
Job Satisfaction and Work Engagement. \textbf{Social Sciences}, v. 7,
n.~6, p.~99, 19 jun. 2018.

NIMALATHASAN, B. Determinants of Key Performance Indicators (KPIs) of
Private Sector Banks in Srilanka: An Application of Exploratory Factor
Analysis. v. 9, n.~2, 2009.

SCHMALTZ, C.; LUEG, R.; AGERHOLM, J. Value-Based Management in Banking:
The Effects on Shareholder Returns. {[}s.d.{]}.

WU, H.-Y. Constructing a Strategy Map for Banking Institutions with Key
Performance Indicators of the Balanced Scorecard. \textbf{Evaluation and
Program Planning}, v. 35, n.~3, p.~303--320, ago. 2012.
\end{frame}



\end{document}
